\section{Time Series Classification}
This section includes a brief introduction to methods which are commonly used in time series classification(TSC). A general pipeline in TSC is comprised of x steps. These are 1. Pre-processing, 2. Feature Extraction, 3. Classification. In some cases such as deep learning, feature extraction and classification are performed simultaneously. 

\subsection{Pre-processing for TSC}

Here we cover 5 methods of pre-processing variable length sequences for TSC. An overview of all the methods is described in the image below.

\import{images/}{pre_processing.tex}
\import{images/}{pre_processing_results.tex}

Given our unbalanced dataset, there is also a need to deal with the unbalanced classes. 

\subsubsection{Truncation / Padding}
The most naive method to transform sequences to equal length is to either truncate or pad the sequence with dummy values. Truncation is performed by discarding the beginning or the end of the sequence to a shorter desired length. On the other hand, padding will add a dummy values to the beginning or the end of the sequence to a new longer length. The two methods can be used jointly as well. For example, say there are variable length sequences with a lengths that vary from $N$ to $M$ and we want a fixed length of $L$. We can truncate sequences with a length larger $L$ or more and pad sequences with a length less than $L$.

\subsubsection{Resampling through interpolation / Extrapolation}
Firstly, the difference between interpolation and extrapolation should be clarified. Interpolation is to fill gaps in between the given data points, whereas extrapolation is to predict future values from the given data points. 
Although there are many methods to perform interpolation and extrapolation, the physical bounds that arise from nature of the data used in this study makes the Kalman filter a fitting method. 

\subsubsection{Randomized Time Warping}
Randomised time warping(RTW) is a randomised extension of dynamic time warping(DTW) for motion recognition\cite{suryanto2016randomized}. RTW generates time elastic (TE) features by randomly sampling the sequential data while retaining the temporal information. A set of TE features is represented by a low-dimensional subspace, called the sequence hypothesis (Hypo) subspace, and the similarity between two sequential patterns is defined by the canonical angles between the two corresponding Hypo subspaces. In essence, RTW simultaneously computes multiple degrees of similarities between a number of warped patterns’ pair candidates, while in practice, RTW generalizes the Hankel matrix commonly used in modeling of system dynamics. 

% \subsubsection{Symbolic Aggregate Approximation}
% Symbolic Aggregate approximation (SAX) is a classical symbolic approach for time series data mining, the basic concept of which is to convert the numerical form of a time series into a sequence of discrete symbols according to designated mapping rules. SAX can reduce dimensionality/ numerosity of data and has a lower bound to the Euclidean distance, that is, the error between the distance in the SAX representation and the Euclidean distance in the original data is bounded.

\subsubsection{Under-Bagging}
Diversity Exploration and Negative Correlation Learning on Imbalanced Data Sets.

\subsection{Feature Extraction for TSC}

\subsection{classification for TSC}
One of the most popular and traditional TSC approaches is the use of a nearest neighbor (NN) classifier coupled with a distance function \cite{bagnall2017great}. Particularly, the Dynamic Time Warping (DTW) distance when used with a NN classifier has been shown to be a very strong baseline\cite{lines2015time}.

\subsubsection{K-Nearest Neighbors + Dynamic Time Warping}

\subsubsection{Hierarchical Vote Collective of Transformation-based Ensembles}

\subsubsection{Time Warped ARMA models}

\subsubsection{Enhanced Grassmann Discriminant Analysis with Randomized Time Warping}

\subsubsection{Graph Convolutional Networks}

An Attention Enhanced Graph Convolutional LSTM Network for Skeleton-Based Action Recognition
Spatial Temporal Graph Convolutional Networks for Skeleton-Based Action Recognition

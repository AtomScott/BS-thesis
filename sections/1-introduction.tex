\section{Overview}
A long standing goal in the field of classification is to develop effective techniques which are applicable to real life problems. With recent advancements in motion tracking technology, significant strides towards this goal has been made due to the increased ability to capture richer and more fine-grained human motion data. In parallel, exponential growth in computer processing power has led to an incredible improvement in machine learning methods. For example there have been multiple breakthroughs in object detection, voice recognition and language modeling etc. However, despite this progress, there are relatively few studies which effectively couple the advancements of these technologies to human motion analysis. 

There are a wide range of benefits can be received by progressing research in human motion analysis. One example we find could significantly benefit is the prevention of injury in sport. For professional athletes, sustaining an injury directly leads to an economic loss. For elite young athletes an injury has the potential to squander hopes and dreams of playing as a professional. For societies with healthcare, an injury for an can represent a significant economic cost to society, with studies showing the mean medical cost of a high school varsity athlete was \$709 per injury, \$2223 per injury in human capital costs, and \$10432 per injury in comprehensive costs \cite{knowles2007cost}.

Although there are many studies on the topic of injury prevention, these usually are from either a rigorously medical perspective \cite{myer2004rationale} or a analysis made with a focus on machine learning with empirical (big) data \cite{alderson2015markerless}. This is a sizeable gap in which we try to bridge. 

In this study, we aim to find factors that differentiate the motion between athletes who are at risk of injury. In order to do so, we develop models and techniques that allow us to analyse granular motion data of a well studied sensori-motor control task, the single-leg drop jump (SDJ). 

We demonstrate that the methods introduced can be used to to detect anomalous motions. Moreover, we show that by interpreting the decision mechanism that led to the results can provide valuable practical information.

\section{Contributions}
In summary our research makes the following contributions,
\begin{enumerate}
    \item \textbf{A Method to prevent injuries with computer vision and machine learning} \\ 
    Explanation.
    
    \item \textbf{A framework that returns actionable intelligence as feedback that can be utilised to improve the subjects movement}
    
    \item \textbf{Novel research that intersects injury prevention, motion capture and machine learning.}
\end{enumerate}

\section{Structure}
The structure of this thesis is as follows; In chapter 2 we explain previous research on injury mechanisms in sport and methods for re

% A reported x\% of the population who enjoy sport has sustained an injury at one point of their life. 

% The time window between landing of a foot and ankle sprain is reported to be less than 200ms. Several other injuries also occur within a given a very short time frame and therefore it is impossible for another person to intervene to prevent such an event. This demonstrates how important it is to  be able to prevent such an event from occurring in the first place.

% With the advancements in computer vision and time-series analysis of 3D objects, the possibilities of human oriented applications has gathered interest. It can be used to prevent injuries.

% In this paper we experiment with several time-series analysis models on 3D human pose data and additional force plate data to detect injury inducing motion sequences and propose actionable intelligence to improve such motion sequences.

% 足関節捻挫などのスポーツ外傷は片脚着地動作時に,関節動態を制御できずに,靭帯に過度な負担を強いる
% ことで発生する.しかしながら,足関節捻挫は着地後~200ms といった短時間で生じることが明らかになって
% おり,この短時間に関節動態を適切に制御することは難しい.そのため,受傷前に危険な動作が癖になってい
% るアスリートを発見し,動作改善を行う必要がある.

% You can't write a good introduction until you know what the body of the paper says. Consider writing the introductory section(s) after you have completed the rest of the paper, rather than before.
% Be sure to include a hook at the beginning of the introduction. This is a statement of something sufficiently interesting to motivate your reader to read the rest of the paper, it is an important/interesting scientific problem that your paper either solves or addresses. You should draw the reader in and make them want to read the rest of the paper.

% The next paragraphs in the introduction should cite previous research in this area. It should cite those who had the idea or ideas first, and should also cite those who have done the most recent and relevant work. You should then go on to explain why more work was necessary (your work, of course.)
 
% What else belongs in the introductory section(s) of your paper? 
% A statement of the goal of the paper: why the study was undertaken, or why the paper was written. Do not repeat the abstract. 
% Sufficient background information to allow the reader to understand the context and significance of the question you are trying to address. 
% Proper acknowledgement of the previous work on which you are building. Sufficient references such that a reader could, by going to the library, achieve a sophisticated understanding of the context and significance of the question.
% The introduction should be focused on the thesis question(s).  All cited work should be directly relevent to the goals of the thesis.  This is not a place to summarize everything you have ever read on a subject.
% Explain the scope of your work, what will and will not be included. 
% A verbal "road map" or verbal "table of contents" guiding the reader to what lies ahead. 
% Is it obvious where introductory material ("old stuff") ends and your contribution ("new stuff") begins? 
% Remember that this is not a review paper. We are looking for original work and interpretation/analysis by you. Break up the introduction section into logical segments by using subheads. 

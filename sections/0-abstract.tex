A long standing goal in the field of classification is to develop effective techniques which are applicable to real life problems. With recent advancements in computer vision, significant strides towards this goal has been made due to the increased ability to capture richer and more fine-grained human motion data. For example, it is possible to capture three dimensional motion at a frame rate of 120 frames per second with little error using motion capture technology, as we demonstrate in domain of injury prevention. 

In parallel, exponential growth in computer processing power has led to an incredible improvement in machine learning methods. For example there have been multiple breakthroughs in object detection, voice recognition and language modeling etc. Using pose estimators based on deep learning, it is now possible to extract three dimension motion data from video with the need of motion capture. Thus allowing a lightweight and easy-to-use application of motion analysis in enviroments where the setup cost of motion capture makes the use of it unrealistic. 

Despite this progress, there are relatively few studies which effectively couple the advancements of machine learning technologies to human motion analysis. Therefore in this thesis, two studies that demonstrate the use of motion analysis and machine learning is introduced.

In the first study, we search for factors that differentiate the motion between athletes who are at risk of injury using motion capture and ground reaction force data of a commonly used sensorimotor performance indicator known as the single-leg drop jump.

In the second study, we propose an analysis framework for extracting important information from video and numerical data in fencing matches to assist experts in their analysis work. 

We show that our approaches can provide useful information to assist those working in the domain of human motion analysis. 
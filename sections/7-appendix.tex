% \chapter{Details regarding data}

% \section{Details on the given data}

% % Please add the following required packages to your document preamble:
% % \usepackage[normalem]{ulem}
% % \useunder{\uline}{\ul}{}
% \begin{table}[]
% \begin{tabular}{|c|l|}
% \hline
% {\ul \textbf{Index}} & \multicolumn{1}{c|}{{\ul \textbf{Name}}} \\ \hline
% 0                    & Cervical vertebra 7                      \\ \hline
% 1                    & Calcaneal tuberosity                     \\ \hline
% 2                    & Fibula Head                              \\ \hline
% 3                    & Greater trochanter                       \\ \hline
% 4                    & Suprasternal notch                       \\ \hline
% 5                    & Anterior superior iliac spine (L)        \\ \hline
% 6                    & lateral epicondyle                       \\ \hline
% 7                    & medial malleolus                         \\ \hline
% 8                    & Posterior superior iliac spine (L)       \\ \hline
% 9                    & medial condyle                           \\ \hline
% 10                   & medial epicondyle                        \\ \hline
% 11                   & lateral malleolus                        \\ \hline
% 12                   & Base of first metatarsal bone            \\ \hline
% 13                   & Base of second metatarsal bone           \\ \hline
% 14                   & Base of fifth metatarsal bone            \\ \hline
% 15                   & Head of first metatarsal bone            \\ \hline
% 16                   & Head of second metatarsal bone           \\ \hline
% 17                   & Head of fifth metatarsal bone            \\ \hline
% 18                   & Fibular trochlea of calcaneus            \\ \hline
% 19                   & First distal phalanges                   \\ \hline
% 20                   & Anterior superior iliac spine (R)        \\ \hline
% 21                   & Posterior superior iliac spine (L)       \\ \hline
% 22                   & Acromion (L)                             \\ \hline
% 23                   & Acromion (R)                             \\ \hline
% 24                   & Sustentaculum tali                       \\ \hline
% 25                   & Thoracic vertebrae 8                     \\ \hline
% 26                   & Tibial tuberosity                        \\ \hline
% 27                   & Scaphoid bone                            \\ \hline
% 28                   & Xiphoid process                          \\ \hline
% \end{tabular}
% \end{table}

% \section{Finding the time of landing}

% \[  t   =   \sqrt{\frac{2h}{g}} \\
%         =   \sqrt{\frac{2*0.4441}{9.8}}\\
%         \approx 0.39 (s)    \]
        
Therefore we find the frame where the average foot position on the y-axis is lowest, with the condition that the frame must be in within 39 frames after the highest point.